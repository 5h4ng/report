\documentclass[UTF8,a4paper,12pt]{ctexart}
\usepackage{amsmath}
\numberwithin{equation}{section}
\allowdisplaybreaks[4]       %多行公式中换页
\usepackage{array}
\usepackage{caption}
\usepackage{amssymb}
\usepackage{tikz}
\usepackage{amsthm}
\usepackage{mathrsfs}
\usepackage{dutchcal}
\usepackage{color}
\usepackage{graphicx}    %插入图片
\usepackage{times}
\usepackage{mathptmx}
\usepackage{fancyhdr} %页眉页脚
\usepackage{booktabs}  %三线表
\usepackage[T1]{fontenc}
\usepackage{enumerate}
\usepackage{physics}
\usepackage{siunitx}
\usepackage[ruled,vlined]{algorithm2e}
\usepackage{subcaption}
\usepackage{bicaption}
\usepackage{appendix}


% 中文字体设置,不设置时默认为linux系统自带的宋体fandol-song
%\usepackage{xeCJK}
%\setCJKmainfont{Noto Serif CJK SC} % 如果有生僻字,可以换用思源宋体为主要字体
%\setCJKsansfont{Noto Sans CJK SC}
%\setCJKmonofont{Noto Sans Mono CJK SC}

% 英文字体设置
%\setmainfont{Times New Roman}  % 默认字体也是Roman字体,可以根据自己喜好设置
%\setsansfont{Arial}            % 默认的无衬线字体跟Arial非常接近,可以根据自己喜好设置


% 参考文献设置
\usepackage[backend=biber,style=gb7714-2015,maxnames=3]{biblatex}
\renewcommand{\bibfont}{\small} % 文献表字号
\setlength{\bibitemsep}{0pt}    % 文献表条目间的间距
\addbibresource{main.bib}       % 导入参考文献数据库


% 页面版心大小
\setlength{\textheight}{22cm}
\setlength{\textwidth}{15cm}

% 页边距设置
\setlength{\voffset}{-1.14cm}
\setlength{\hoffset}{-0.57cm}
%\setlength{\headheight}{14.48167pt} 
\setlength{\headheight}{1cm}
\setlength{\topmargin}{0cm}
%\setlength{\headsep}{2.9cm}
\setlength{\headsep}{1.8cm}
\setlength{\footskip}{1.2cm}


% 页眉页脚设置
\pagestyle{fancy}
\fancyhf{}
\fancyfoot[C]{\thepage}
% 只需要区分fancy和empty页面,每章的页眉页脚需手动定义
\fancypagestyle{plain}{
  \pagestyle{fancy}     % 将plain页面格式替换为fancy,确保目录页有页眉
}
\fancyhfinit{\small} % 页眉页脚字号

% 双线页眉
\makeatletter
\def\headrule{{\if@fancyplain\let\headrulewidth\plainheadrulewidth\fi%
\hrule\@height 1.5pt \@width\headwidth\vskip1.5pt%上面线为1pt粗
\hrule\@height 0.5pt\@width\headwidth  %下面0.5pt粗
\vskip-2\headrulewidth\vskip-1pt}      %两条线的距离1pt
  \vspace{6mm}}     %双线与下面正文之间的垂直间距
\makeatother


% 行距
\usepackage{setspace}
\setlength{\baselineskip}{20pt}
\newcommand*{\circled}[1]{\lower.7ex\hbox{\tikz\draw (0pt, 0pt)%
    circle (.5em) node {\makebox[1em][c]{\small #1}};}}


% 目录设置
\usepackage{hyperref}  
\hypersetup{hidelinks}

\usepackage{tocloft} 
\renewcommand{\cftsecleader}{\cftdotfill{\cftdotsep}} %为目录中section补上引导点
\usepackage{titletoc}
\titlecontents{section}[0pt]
              {\addvspace{6pt}\filright\large\bf} %要将ABSTRACT的字体也替换为Arial的话,在本括号中末尾加上\ttfamily\songti
              {\contentspush{\thecontentslabel \quad }} %
              {}{\titlerule*[8pt]{.}\contentspage}
\setlength{\cftbeforesubsecskip}{6pt}
\setlength{\cftbeforesubsubsecskip}{6pt}

% 目录缩进
\setlength{\cftsubsecindent}{1em}
\setlength{\cftsubsubsecindent}{2em}

% 目录字体
\renewcommand{\cftsubsecfont}{\normalsize}
\renewcommand{\cftsubsubsecfont}{\small}


% 图表编号
\captionsetup[figure][bi-second]{name=Figure} %设置图的英文编号前缀
\captionsetup[table][bi-second]{name=Table} %设置表的英文编号前缀
\numberwithin{equation}{section}%公式按章节编号
\numberwithin{figure}{section}%图表按章节编号
\numberwithin{table}{section}
\renewcommand {\thefigure} {\thesection{}-\arabic{figure}}%设定图片的编号。这样设置的实现效果为图1-1
\renewcommand {\thetable} {\thesection{}-\arabic{table}}


% 图/表标题格式
\captionsetup{font={small,bf},labelsep=quad,justification=centering} 
\captionsetup[subfigure]{labelfont=normalfont,textfont=normalfont} % 子图题不加粗


% 浮动体间距
%\setlength{\intextsep}{6pt}    % h浮动体与上下文间距
%\setlength{\floatsep}{6pt}     % 浮动体之间的间距
%\setlength{\textfloatsep}{6pt} % t/b浮动体与正文邻接间距


% 表内字体
\usepackage[captionskip=6pt]{floatrow}
\floatsetup[table]{font={small},capposition=top}


% 各级标题格式
\ctexset{section={
  format={\heiti \zihao{3} \bfseries \center},
  number={第\chinese{section}章}
}}
\usepackage{titlesec}
\titlespacing*{\section}{0pt}{24pt}{18pt}
\titlespacing{\subsection}{0pt}{24pt}{12pt}
\titlespacing{\subsubsection}{0pt}{12pt}{6pt}
\titleformat*{\subsection}{\heiti\large\bfseries}
\titleformat*{\subsubsection}{\heiti\normalsize\bfseries}


% autoref中文名称
\def\equationautorefname{式}
\def\footnoteautorefname{脚注}
\def\itemautorefname{项}
\def\figureautorefname{图}
\def\tableautorefname{表}
%\def\partautorefname{篇}
\def\appendixautorefname{附录}
%\def\chapterautorefname{章} % 不使用chapter,而使用section作为章
\def\sectionautorefname{} % 由于已经修改章节名称为第X章,应该在autoref中不加前缀
\def\subsectionautorefname{节}
\def\subsubsectionautorefname{小节}
\renewcommand{\algorithmcfname}{算法}
\renewcommand{\algorithmautorefname}{算法}


% 盲审模式控制
\newif \ifreview
%\reviewtrue     %开启盲审模式,反之注释掉
\reviewfalse    %关闭盲审模式


% 打印模式控制(需要在章节划分处使用\clearsection命令)
\newif \ifprint
%\printtrue      %打印模式
\printfalse     %非打印模式,建议用于生成电子版

\ifprint
\newcommand{\clearsection}{\clearpage \ifodd\value{page}\else \thispagestyle{empty}\hbox{}\newpage\fi} % 打印模式下,每章右页起
\else
\newcommand{\clearsection}{\clearpage} % 非打印模式,连续排版
\fi




\begin{document}

\input{sec/0.0-front.tex}

\ifreview
\else
\thispagestyle{empty}
\begin{center}
\heiti \zihao{3}\textbf{
上海交通大学\\
学位论文原创性声明}
\end{center}

\zihao{-4}
本人郑重声明:所呈交的学位论文,是本人在导师的指导下,独立进行研究工作所取得的成果。除文中已经注明引用的内容外,本论文不包含任何其他个人或集体已经发表或撰写过的作品成果。对本文的研究做出重要贡献的个人和集体,均已在文中以明确方式标明。本人完全知晓本声明的法律后果由本人承担。

\begin{flushright}
\begin{tabular}{l}
\zihao{4}
学位论文作者签名:
\begin{minipage}{30mm}
\quad % 电子签名图片
\end{minipage}\\
\zihao{4}
日期:\qquad 年\quad 月\quad 日
\end{tabular}
\end{flushright}

~\\
\begin{center}
\heiti \zihao{3}\textbf{
上海交通大学\\
学位论文使用授权书}
\end{center}

本人同意学校保留并向国家有关部门或机构送交论文的复印件和电子版,允许论文被查阅和借阅。\\
本学位论文属于 :\par
$\square$\textbf{公开论文}\par
%\vspace{-\baselineskip}\textbf{\checkmark}\par %在上一行的方框内打勾
$\square$\textbf{内部论文},保密$\square$1年/$\square$2年/$\square$3年,过保密期后适用本授权书。\par
$\square$\textbf{秘密论文},保密\_\_\_年(不超过10年),过保密期后适用本授权书。\par
$\square$\textbf{机密论文},保密\_\_\_年(不超过20年),过保密期后适用本授权书。\par
(请在以上方框内选择打“\textbf{\checkmark}”)\\

\begin{flushright}
\zihao{4}
\begin{tabular}{l l}
学位论文作者签名:
\begin{minipage}{35mm}
\quad % 电子签名图片
\end{minipage}
&指导教师签名:
\begin{minipage}{22mm}
\quad % 电子签名图片
\end{minipage} \\
日期:\qquad 年\quad 月\quad 日 &日期:\qquad 年\quad 月\quad 日\\
\end{tabular}
\end{flushright}

\clearsection
\fi

\pagenumbering{Roman}
\fancyhead[LH]{上海交通大学学位论文}
\fancyhead[RH]{}

\input{sec/0.2-abstract.tex}

\renewcommand\contentsname{\heiti \textbf{目\quad 录}}

\begin{center}
\tableofcontents
\end{center}

\clearsection

\pagenumbering{arabic}


\fancyhead[LH]{上海交通大学学位论文}
\fancyhead[RH]{第一章\quad 绪论}

\section{绪论}
\subsection{引言}
学位论文……
\subsection{本文主要研究内容}
本文……
\subsection{本文研究意义}
本文……
\subsection{本章小结}
本文……

\clearsection

\input{sec/2-related.tex}

\input{sec/3-framework.tex}

\input{sec/4-summary.tex}


\fancyhead[LH]{上海交通大学学位论文}
\fancyhead[RH]{参考文献}


%\nocite{*}     % 添加未引到的参考文献


% 常见问题:引用网络文献时缺少引用日期。
% 解决方案:使用@online而非@misc,并且添加urldate={20xx-xx-xx}字段

\begin{center}
\setstretch{1.15}
\printbibliography[heading=bibintoc, title={参\quad 考\quad 文\quad 献}]
\end{center}

\clearsection

\fancyhead[LH]{上海交通大学学位论文}
\fancyhead[RH]{附录1}

\addcontentsline{toc}{section}{符号与标记(附录1)}
\section*{符号与标记(附录1)}

\clearsection

\ifreview
\input{sec/-1.2-ouput-review.tex}
\else
\input{sec/-1.2-ouput.tex}
\fi

\ifreview
\else
\input{Latex-UG-SJTU/sec/-1.3-acknowledgements}
\fi

\pagenumbering{arabic}
\fancyhead[LH]{上海交通大学学位论文}
\fancyhead[RH]{}
\section*{NUMERICAL SIMULATION OF HOMOGENEOUS CHARGE COMPRESSION IGNITION COMBUSTION FUELED WITH DIMETHYL ETHER}%英文大摘要标题

\hspace{8mm}HCCI (Homogenous Charge Compression Ignition) combustion has advantages in terms of efficiency and reduced emission. HCCI combustion can not only ensure both the high economic and dynamic quality of the engine, but also efficiently reduce the NOx and smoke emission. Moreover, one of the remarkable characteristics of HCCI combustion is that the ignition and combustion process are controlled by the chemical kinetics, so the HCCI ignition time can vary significantly with the changes of engine configuration parameters and operating conditions. ……%(英文大摘要正文)

\clearsection


\end{document} 