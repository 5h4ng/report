\documentclass[UTF8]{ctexart}

\usepackage{amsmath}
\usepackage{cases}
\usepackage{cite}
\usepackage{graphicx}
\usepackage[margin=1in]{geometry}
\geometry{a4paper}
\usepackage{fancyhdr}
\usepackage{booktabs}
\usepackage{float}
\pagestyle{fancy}
\fancyhf{}


\title{牛乳中酪蛋白的制备}
\author{522111910161 尚子翔}
\date{\today}
\pagenumbering{arabic}

\begin{document}

\fancyhead[L]{}
\fancyhead[C]{牛乳中酪蛋白的制备}
\fancyfoot[C]{\thepage}

\maketitle
\tableofcontents
\newpage

\section{实验目的}
\begin{enumerate}
    \item 学习从牛乳中分离纯化酪蛋白的原理和方法。
    \item 掌握等电点沉淀的原理。
\end{enumerate}



\section{实验原理}
\subsection{牛乳中的蛋白质} 
    牛乳中主要含有酪蛋白和乳清蛋白两种蛋白质。其中酪蛋白占了牛乳蛋白质的80\%,酪蛋白是白色、无味的物质,不溶于水、乙醇及有机溶剂,但溶于碱溶液。牛乳在pH4.7时酪蛋白等电聚沉后剩余的蛋白质统称乳清蛋白。乳清蛋白不同于酪蛋白,其粒子的水合能力强、分散性高,在乳中呈高分子状态。
\subsection{等电点沉淀原理}
    等电点沉淀是一种分离和纯化蛋白质的常用方法。蛋白质是由氨基酸组成的复杂有机分子,具有带电性质。在特定pH条件下,蛋白质中的氨基酸会带正电荷或负电荷。蛋白质溶液的pH值会影响其电荷性质,使得蛋白质在不同pH条件下带有不同的净电荷。

等电点是指在特定条件下,蛋白质的净电荷为零的pH值。当溶液的pH等于蛋白质的等电点时,蛋白质的带电离子与其周围的溶剂分子相互吸引,导致蛋白质失去水合层,从而凝聚成聚集体或沉淀。这是因为在等电点时,蛋白质带电离子相互中和,失去了对水分子的亲和力,从而失去溶解性,形成沉淀。

因此,通过调节溶液的pH值接近蛋白质的等电点,可以使蛋白质沉淀出来,从而实现蛋白质的分离和提取。这种原理在蛋白质分离纯化、酶学研究、生物化学实验等领域得到广泛应用。

\subsection{实验综述}
本法利用等电点时溶解度最低的原理,将牛乳的pH调至4.7时,酪蛋白就沉淀出来,用乙醇洗涤沉淀物,除去脂类杂质后便可得到纯的酪蛋白。酪蛋白含量约为35g/L。


\section{方案}

\subsection{试剂和设备}
\begin{itemize}
    \item 试剂 \\
    95\%乙醇,无水乙醚,乙醇-乙醚混合液(V/V=1:1),冰乙酸。\\
    0.2mol/L醋酸-醋酸钠缓冲液:\\
A液:称取 $NaAc\cdot 3H_2O$ 54.44g,定容至2000 mL。\\
B液:称取优级纯醋酸(含量大于99.8\%)12.0g,定容至1000mL.
    \item 材料\\
    牛奶制品。
    \item 器材\\
    低速离心机,精密pH试纸或酸度计,水浴锅,烧杯,温度计,天平。
\end{itemize}

\subsection{实验步骤与操作}
\begin{enumerate}
    \item 将 20 毫升(20 克)牛奶放入 125 毫升烧瓶中,在 40 摄氏度水浴中加热。
    \item 边搅拌边缓慢加入 20 毫升醋酸缓冲液。注意加酸速度不要太快。用精确的试纸或 pH 计将 pH 值调至 4.7。
    \item 离心 15 分钟(3000r/min)。倾析上清液,得到酪蛋白粗品。
    \item 用等体积的水洗涤沉淀两次。离心 15 分钟(3000r/min), 并倾析上清液。
    \item 在沉淀中加入 10 毫升 95\% 的乙醇,充分搅拌以破碎产物。离心 10 分钟(3000r/min),倾去上清液。
    \item 分别用 1:1 的乙醚-乙醇混合物和乙醚洗涤沉淀。
    \item 让固体充分沥干,然后将其刮入称重的 滤纸,让其在烘箱中干燥。  
    \item 按以下方法计算牛奶中的酪蛋白百分比:\\
    \% Casein =  (grams of casein \ grams of milk) x 100
\end{enumerate}
\subsection{注意事项}
\begin{enumerate}
    \item 该实验利用等电点沉淀原理,所以调pH值的步骤一定要精准。
    \item 实验室保证实验室空气流畅,以防被乙醚麻醉。
    \item 离心机使用的时候一定要保证EP管摆放平衡且对称。
\end{enumerate}
\section{实验数据记录和处理}
\subsection{离心后的样品}
\begin{figure}[h]
    \centering
    \includegraphics[width=0.6\textwidth]{微信图片_20231021041100.jpg}
    \caption{离心后的样品}
    \label{fig:enter-label}
\end{figure}
\subsection{实验结果}
\begin{figure}[h]
    \centering
    \includegraphics[width=0.6\textwidth]{result.jpg}
    \caption{Caption}
    \label{fig:enter-label}
\end{figure}

\section{数据分析讨论、总结建议}
\subsection{分析讨论}
\begin{enumerate}
    \item 第一次离心过后,EP管中应该呈现明显分层。上层清液澄澈微微发黄,里面的成分主要是些脂类、乳清蛋白、乳糖以及少量的水溶性物质;下层沉淀呈小颗粒状,主要成分是凝聚成团的酪蛋白。总体来讲,离心后的样品非常理想,保证后续步骤能都得到比较纯净的酪蛋白样品,这也说明上一步调节pH比较精确,沉淀完全。
    \item 烘干后的的酪蛋白颗粒微微发黄,颗粒比较细。细小粉末状的形态说明在前一步吸水研磨做的比较好,最后样品中几乎没有水分杂志。微微发黄的原因可能是蛋白中的氨基酸和脂质成分在暴露在空气中时可能会发生氧化反应,导致颜色的微弱变化;也有可能是在加热处理的时候导致酪蛋变性,呈现微微的黄色;最主要的原因可能是酪蛋白中可能含有一些杂质,如脂质或其他附着在蛋白质表面的物质,这些杂质可能会使酪蛋白呈现微弱的黄色或其他颜色。
    \item 最终测得酪蛋白含量为\[\% Casein =  \frac{grams\ of\ casein}{grams\ of\ milk} \times 100 = \frac{0.628g}{20g} \times 100 = 3.125\]所选用的样品中蛋白质含量为0.72g/20ml,而酪蛋白大约占牛奶中蛋白质的80\%,所以理论上应该测得酪蛋白0.56g,相对误差为12.14\%,和其他组的实验结果相比,是比较准确误差较小的。结果偏大的原因主要是因为酪蛋白样品中的杂质还没有完全清楚,比如说水和脂质。
\end{enumerate}
\subsection{总结建议}
\begin{enumerate}
    \item 在调节牛奶-缓冲液体系的pH时,要注意精密pH试纸的比色卡上由于有一层透明膜,所以颜色会显得较浅,4.7pH应该对应比色卡上4.4pH处的黄绿色。
    \item 在进行第一次离心时,可以先将样品离心2分钟,观察分层是否明显,若分层明显说明前几步操作没有出现问题,可以继续离心13分钟使酪蛋白完全沉淀;若分离不明显可以节省时间直接重新实验。
    \item 在洗涤过程中,每次离心后都需要将EP管中的样品在洗涤液中重悬,具体方法是:使用移液枪的活塞来反复吸取和排出溶液,以便将样品彻底悬浮于液体中。反复吸取和排出可促进样品的均匀分散。
    \item 在将洗涤后的样品放入烘箱前要用干净滤纸反复吸水,尽可能使样品干燥,然后将样品颗粒研磨。这样可以保证样品在烘箱内快速均匀受热,不会形成外面干燥焦黄,内部潮湿的大颗粒酪蛋白。放入烘箱内的时间也不宜过长,使酪蛋白焦化变性。
\end{enumerate}
\section{思考题}
\subsection{制备高产率纯酪蛋白的关键是什么?}
\begin{enumerate}
    \item 优化提取方法: 选择合适的提取方法以提高酪蛋白的产量和纯度,如等电点沉淀法、离心沉淀法或色谱分离法等。确保选择的方法可以最大程度地提高酪蛋白的提取效率。
    \item 精确控制条件: 在提取过程中,精确控制各项条件如温度、pH值、离心速度等,确保在最适合的条件下进行提取,从而提高产率和纯度。
    \item 杂质去除: 在提取过程中,及时清除可能的杂质,如脂质、乳糖等,以保证得到的酪蛋白样品纯度较高。
\end{enumerate}
\end{document}
